\documentclass{article}
\usepackage{enumerate}
\usepackage{amsmath}
\usepackage{amssymb}
\usepackage{graphicx}
\usepackage{subfigure}
\usepackage{geometry}
\usepackage{caption}
\usepackage{indentfirst}
\usepackage[colorlinks,urlcolor=blue]{hyperref}
\usepackage{minted}
\usemintedstyle{autumn}
\setminted{linenos,breaklines,tabsize=4,xleftmargin=1.5em}
\geometry{left=3.0cm,right=3.0cm,top=3.0cm,bottom=4.0cm}
\title{VE482 Lab 4}
\author{Liu Yihao 515370910207}
\date{}

\begin{document}
\maketitle

\section{Layer programming}
\begin{itemize}
\item The program can be divided into three layers, what are they?

The kernel layer, the logic layer and the interface layer.
\item Split the program into files according to the defined layers.

The kernel layer in list.c/.h, the logic layer in logic.c/.h and the interface layer in interface.c/.h.

\item Create the appropriate corresponding header files.

list.h
\inputminted{c}{list.h}

logic.h
\inputminted{c}{logic.h}

\item If necessary rewrite functions such that no call is emitted from lower level functions to upper level
functions.

list.c
\inputminted{c}{list.c}

logic.c
\inputminted{c}{logic.c}

\item
The initial program implements a command line interface, write a ``Menu interface'' which (i) wel-
comes the user, (ii) prompts him for some task to perform, and (iii) runs it. When a task is
completed the user should (i) be informed if it was successful and then (ii) be displayed the menu.
From the menu he should be able to exit the program.

\item
Write two main functions, one which will ``dispatch'' the work to another function which will run
the command line user interface and a second one which will ``dispatch'' the work to the Menu user
interface.

interface.h
\inputminted{c}{interface.h}

interface.c
\inputminted{c}{interface.c}

ui.c
\inputminted{c}{ui.c}

cli.c
\inputminted{c}{cli.c}
\end{itemize}

\section{Libraries}
\begin{itemize}
\item What are the three stages performed when compiling a file?

Preprocess, compilation and link.

\item Briefly describe each of them.
\begin{itemize}
\item Preprocess: substitute definitions and proceed with pragmas
\item Compilation: compile source code into binary files and generate symbol table
\item Link: link binary files according symbol table
\end{itemize}

\item Search more details on how to proceed.

It's very easy to use CMake to generate static libraries with the ``add\_library'' command.

\item Create two static libraries, one for each of the two lowest layers in the previous program.
\begin{minted}{cmake}
add_library(l4_list_static STATIC list.c)
add_library(l4_logic_static STATIC logic.c)
\end{minted}

\item Compile the command line version of the program using these two static libraries.
\begin{minted}{cmake}
add_executable(l4_cli_static cli.c interface.c)
target_link_libraries(l4_cli_static l4_logic_static l4_list_static)
\end{minted}

\item Generate two dynamic libraries, one for each of the two lowest layers in the previous program.
\begin{minted}{cmake}
add_library(l4_list_dynamic SHARED list.c)
add_library(l4_logic_dynamic SHARED logic.c)
target_link_libraries(l4_logic_dynamic l4_list_dynamic)
\end{minted}

\item Compile the whole program
\begin{minted}{cmake}
add_executable(l4_cli cli.c interface.c logic.c list.c)
add_executable(l4_ui ui.c interface.c logic.c list.c)
\end{minted}

\item Compile the Menu version of the program using these two dynamic libraries.
\begin{minted}{cmake}
add_executable(l4_ui_dynamic ui.c interface.c)
target_link_libraries(l4_ui_dynamic l4_api_dynamic l4_list_dynamic)
\end{minted}

\end{itemize}



\end{document}
