\documentclass{article}
\usepackage{enumerate}
\usepackage{amsmath}
\usepackage{amssymb}
\usepackage{graphicx}
\usepackage{subfigure}
\usepackage{geometry}
\usepackage{caption}
\usepackage{indentfirst}
\usepackage[colorlinks,urlcolor=blue]{hyperref}
\usepackage{minted}
\usemintedstyle{autumn}
\setminted{linenos,breaklines,tabsize=4,xleftmargin=1.5em}
\geometry{left=3.0cm,right=3.0cm,top=3.0cm,bottom=4.0cm}
\title{VE482 Lab 2}
\author{Liu Yihao 515370910207}
\date{}

\begin{document}
\maketitle

\section{Basic shell}
\begin{itemize}
\item Use the \mintinline{c}{mkdir}, \mintinline{c}{touch}, \mintinline{c}{mv}, \mintinline{c}{cp}, and \mintinline{c}{ls} commands to:
\begin{itemize}
\item Create a file named \mintinline{c}{test}.
\begin{minted}{shell}
touch test
\end{minted}
\item Move test to \mintinline{c}{dir/test.txt}, where \mintinline{c}{dir} is a new directory.
\begin{minted}{shell}
mkdir dir
mv test dir/test.txt
\end{minted}
\item Copy \mintinline{c}{dir/test.txt} to \mintinline{c}{dir/test_copy.txt}.
\begin{minted}{shell}
cp dir/test.txt dir/text_copy.txt
\end{minted}
\item List all the files contained in \mintinline{c}{dir}.
\begin{minted}{shell}
ls dir -a
\end{minted}
\end{itemize}

\item Use the \mintinline{c}{grep} command to:
\begin{itemize}
\item List all the files form \mintinline{c}{/etc} containing the pattern \mintinline{latex}{127.0.0.1}.
\begin{minted}{shell}
grep -rl '127.0.0.1' /etc
\end{minted}
\item Only print the lines containing your username and root in the file /etc/passwd (only one grep should be used)
\begin{minted}{shell}
grep -rE '(liu|root)' /etc/passwd
\end{minted}
\end{itemize}

\item Use the \mintinline{c}{find} command to:
\begin{itemize}
\item List all the files from \mintinline{c}{/etc} that have been accessed less than 24 hours ago.
\begin{minted}{shell}
find /etc -atime 1
\end{minted}
\item List all the files from \mintinline{c}{/etc} whose name contains the pattern ``netw''.
\begin{minted}{shell}
find /etc -name '*netw*'
\end{minted}
\end{itemize}

\item In the bash man-page read the part related to redirections. Explain the following signs \mintinline{c}{>}, \mintinline{c}{>>}, \mintinline{c}{<<<}, \mintinline{c}{>&1}, and \mintinline{c}{2>&1 >}. What is the use of the tee command. \\[0.5em]
\mintinline{c}{>} redirects the standard output into a file. \\[0.5em]
\mintinline{c}{>>} redirects and appends the standard output into a file. \\[0.5em]
\mintinline{c}{<<<} redirects the contents on the right as the standard input of the command on the left. \\[0.5em]
\mintinline{c}{>&1} redirects the standard output into standard output (meaningless). \\[0.5em]
\mintinline{c}{2>&1 >} redirects the standard error into standard output, and redirects the origin standard output into a file.

\item Explain the behaviour of the \mintinline{c}{xargs} command and of the | sign. \\[0.5em]
\mintinline{c}{xargs} is used to build and execute command lines from standard input, by combining multi lines and extra spaces into a line with single spaces.\\[0.5em]
The | sign pipes the standard output of the command on the left into the command on the right as the standard input.

\item What are the \mintinline{c}{head} and \mintinline{c}{tail} commands? How to ``live display'' a file as new lines are appended? \\[0.5em]
\mintinline{c}{head} and \mintinline{c}{tail} are used to get the first and last several lines of a file. \\[0.5em]
Use the \mintinline{c}{-f} option of \mintinline{c}{tail} to ``live display'' a file as new lines are appended.

\item How to monitor the system using \mintinline{c}{ps}, \mintinline{c}{top}, \mintinline{c}{free}, \mintinline{c}{vmstat}? \\[0.5em]
\mintinline{c}{ps} is used to monitor the processes. \\[0.5em]
\mintinline{c}{top} is used to monitor the CPU and RAM of processes. \\[0.5em]
\mintinline{c}{free} is used to monitor the RAM. \\[0.5em]
\mintinline{c}{vmstat} is used to monitor the RAM, IO and CPU in a period. 

\item In Minix 3, how to manage softwares (install, remove, update... )?
\begin{minted}{shell}
pkgin update       # Update the package repository
pkgin install name # Install a package
pkgin remove name  # Remove a package
pkgin upgrade name # Upgrade a package
pkgin search name  # Search a package
\end{minted}

\item What is the purpose of the commands \mintinline{c}{ifconfig}, \mintinline{c}{adduser}, and \mintinline{c}{passwd}?\\[0.5em]
\mintinline{c}{ifconfig} is used to check the state of network. \\[0.5em]
\mintinline{c}{adduser} is used to create a new user. \\[0.5em]
\mintinline{c}{passwd} is used to set password for the current user. 
\end{itemize}

\section{Working on a remote server}
\begin{itemize}
\item Setup an SSH server on Minix 3. From Linux (using ssh) or Windows (using Putty) log into Minix 3. Note: the network need to be properly setup on the Virtual Machine (VM).
\begin{minted}{shell}
ssh root@localhost
\end{minted}

\item What is the default SSH port? Change this port for port 2222. Log into Minix 3 using this new SSH server setup. \\[0.5em]
The default port is 22.\\[0.5em]
On Minix3:
\begin{minted}{shell}
vi /etc/ssh/sshd_config
\end{minted}
and edit the option ``Port''. \\[0.5em]
On Linux:
\begin{minted}{shell}
ssh root@localhost -p2222
\end{minted}

\item List and explain the role of each the file in the \mintinline{shell}{$HOME/.ssh} directory. In \mintinline{shell}{$HOME/.ssh/config}, create an entry for Minix 3.

\begin{minted}{shell}
ls $HOME/.ssh
\end{minted}
%$
shows the files
\begin{minted}{shell}
config  id_rsa  id_rsa.pub  known_hosts
\end{minted}

\mintinline{c}{config} is the config file of ssh client. \\[0.5em]
\mintinline{c}{id_rsa} is the private key of ssh client/server. \\[0.5em]
\mintinline{c}{id_rsa.pub} is the public key of ssh client/server. \\[0.5em]
\mintinline{c}{known_hosts} is the ssh servers logged before.


On Linux:
\begin{minted}{shell}
vi ~/.ssh/config
\end{minted}
and add these lines
\begin{minted}{shell}
Host    minix
    HostName        localhost
    Port            2222
    User            root
\end{minted}
Then directly connect Minix 3 by
\begin{minted}{shell}
ssh minix
\end{minted}

\item Briefly explain how key-only authentication works in SSH. Generate a key-pair on the host system and use it to log into Minix 3 without a password.\\[0.5em]
First Alice (ssh client) generates a pair of key, and sends the public key to Bob (ssh server). Bob adds the key to a list of authorized hosts. When Alice wants to login, she sends the public key again and Bob uses principles of public key cryptography to verify the identity of Alice. If success, Bob and Alice can connect with a tunnel encrypted by their key pairs.

\begin{minted}{shell}
ssh-keygen -t rsa
scp -P2222 ~/.ssh/id_rsa.pub root@localhost:/root
ssh minix
cat ~/id_rsa.pub >> ~/.ssh/authorized_keys
exit
ssh minix -i ~/.ssh/id_rsa
\end{minted}

\end{itemize}

\section{Basic Bash scripting}
\begin{itemize}
\item What should be the first line of a Bash script?
\begin{minted}{shell}
#/bin/bash
\end{minted}
\item What are the main differences between \mintinline{c}{sh}, \mintinline{c}{bash}, \mintinline{c}{csh}, and \mintinline{c}{zsh}? \\[0.5em]
\mintinline{c}{sh} is the origin shell of Unix, \mintinline{c}{bash}, \mintinline{c}{csh}, and \mintinline{c}{zsh} are different expansion of \mintinline{c}{sh}, they are all compatible with \mintinline{c}{sh}, but perhaps not compatible with each other.

\item How to define and access variables?
\begin{minted}{shell}
var=1         # define a variable named var and assign it as 1
let var=var+1 # maths calculation
var=$var+1    # string concat
echo ${var}   # echo the variable
\end{minted}

\item What is the meaning of \mintinline{shell}{$0, $1,..., $?, $!}? \\[0.5em]
\mintinline{shell}{$0} means \mintinline{c}{argv[0]} in C. \\[0.5em]
\mintinline{shell}{$1} means \mintinline{c}{argv[1]} in C. \\[0.5em]
\mintinline{shell}{$?} means the exit status of the last command. \\[0.5em]
\mintinline{shell}{$!} means the process id of the last command.

\item How to define arrays and access or assign elements?
\begin{minted}{shell}
array=(var1 var2 var3 varN)
echo ${array[0]}
array[3]=var3
echo ${array[3]}
\end{minted}

\item How to perform \mintinline{shell}{if} and \mintinline{shell}{switch} statements? Provide an example.
\begin{minted}{shell}
# if statement
if [ "$1" = "" ]; then
    echo 0
elif [ "$1" = 1 ]; then
    echo 1
else
    echo 2
fi

# switch statement
case $1 in 
    1 )
        echo 1 ;; 
    2 )
        echo 2 ;; 
    * ) 
        echo 0 ;;
esac
\end{minted}
%$

\item What are the various syntaxes of a \mintinline{shell}{for} loop? For each type write a sample code.

\begin{minted}{shell}

# for-in loop
for file in * ; do  
    echo $file  
done

# c-style for loop
for ((i=0; i<10; ++i)); do
    echo $i
done

\end{minted}

\item How to write a \mintinline{shell}{while} loop?
\begin{minted}{shell}
i=0
while [ $i -lt 10 ]; do  
    echo $i  
    let i=i+1
done
\end{minted}

\item What is the use of the \mintinline{c}{PS3} variable? Provide a short code example. \\[0.5em]
PS3 is the select prompt \mintinline{latex}{#?}, it is displayed in a select statement.

\begin{minted}{shell}
echo "Who is the strongest person in JI"
choices='xd qss gg xyy wgz' 
select person in $choices; do 
    if [ $person ]; then 
        echo "$person is the strongest person in JI." 
        break
    else 
        echo "Invalid, select again." 
    fi 
done
\end{minted}
%$

\item What is the purpose of the \mintinline{c}{iconv} command, and why is it useful? \\[0.5em]
The \mintinline{c}{iconv} command reads in text in one encoding and outputs the text in another encoding. It is useful because there are various encoding in different operation systems, if one wants to use a file from another environment, the transform is important.

\item Given a variable \mintinline{shell}{$temp} what is the effect of \mintinline{shell}{${#temp}}, \mintinline{shell}{${temp%%word}}, \mintinline{shell}{${temp/pattern/string}}. \\[0.5em]
\mintinline{shell}{${#temp}} means the length of \mintinline{shell}{$temp}. \\[0.5em]
\mintinline{shell}{${temp%%word}} deletes \mintinline{shell}{word} on the right of \mintinline{shell}{$temp}. \\[0.5em]
\mintinline{shell}{${temp/pattern/string}} replaces \mintinline{shell}{pattern} with \mintinline{shell}{string} in \mintinline{shell}{$temp}.

\item Search what are ``regular expressions" and how to use them in a \mintinline{c}{grep} or \mintinline{c}{find} command. Give some simple examples based on files and keywords used in exercise 2 of assignment 2.\\[0.5em]
Regular expressions are some very useful expressions that I can never understand. However, there are also some masters in RegExp such as xyy, xtt and xd.

\end{itemize}

Two programming languages often used in conjunction with Bash are \mintinline{c}{sed} and \mintinline{c}{awk}.

\begin{itemize}
\item Provide a brief introduction to both of them, explaining how to use them and when they reveal to be the most helpful. \\[0.5em]
Sed  is  a  stream editor.  A stream editor is used to perform basic text transformations on an input stream (a file or input from a pipeline).  While in some ways similar to an
       editor which permits scripted edits (such as ed), sed works by making only one pass over the input(s), and is consequently more efficient.  But it is  sed's  ability  to  filter
       text in a pipeline which particularly distinguishes it from other types of editors. \\[0.5em]
awk scans each input file for lines that match any of a set of patterns
     specified literally in prog or in one or more files specified as -f
     filename.  With each pattern there can be an associated action that will
     be performed when a line of a file matches the pattern.  Each line is
     matched against the pattern portion of every pattern-action statement;
     the associated action is performed for each matched pattern.  The file
     name - means the standard input.  Any file of the form var=value is
     treated as an assignment, not a filename, and is executed at the time it
     would have been opened if it were a filename.

\item Using \mintinline{c}{curl} or \mintinline{c}{wget} retrieve information on \href{http://aqicn.org/?city=Shanghai&widgetscript&size=large}{shanghai air quality} and pipe it to \mintinline{c}{sed} which should parse the output in order to display the information in the terminal following the format below

AQ: value Temp: value $\rm^\circ C$ (e.g. AQ: 55 Temp: 24 $\rm^\circ C$).

\begin{minted}{shell}
curl --silent 'http://aqicn.org/?city=Shanghai&widgetscript&size=large' \
| sed ':t;N;s/\n//;b t' \
| sed 's/\([^\n]\+\)\\\">/AQ: /g' \
| sed "s/<\/div>\([^\n]\+\)10px;'>/ Temp: /g" \
| sed 's/<\/td>\([^\n]\+\)//g' \
&& echo -e '\u00B0C'
\end{minted}

\item Pipelining the output of \mintinline{c}{ifconfig} to \mintinline{c}{awk} return only the ip address of your current active network connection (the active network interface can be passed to \mintinline{c}{ifconfig}).

\begin{minted}{shell}
ifconfig wlp2s0 | awk '{if(NR==1){print $2}}'
\end{minted}
%$

\end{itemize}


\end{document}
